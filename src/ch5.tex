\chapter{数理统计的基础知识}
\section{基本概念}
\subsection{统计量}
\dfn{统计量}{
    设$X_1,X_2,\ldots,X_n$是来自总体$X$的样本, 称此样本的任一不含总体分布\uwave{未知参数}的函数为该样本的\vocab{统计量}.
}
\ex{}{
    设$X\sim N(5,\sigma^2)$, $\sigma$未知. $X_1,X_2,\ldots,X_n$为总体$X$的一个样本,令
    \[S_n = X_1 + X_2 + \cdots + X_n,\ \ \overline{X} = \frac{S_n}{n},\]
    则$S_n$和$\overline{X}$都是样本的统计量. 但$U = \dfrac{n\left( \overline{X} - 5 \right)}{\sigma}$\emph{不是统计量}, 因为它含有\uwave{未知参数$\sigma$}.
}
\subsection{常用统计量}
\subsubsection{样本均值}
\begin{equation}
    \overline{X} = \frac{1}{n}\sum_{i=1}^n X_i
\end{equation}
\subsubsection{样本方差 (有偏, 无偏)}
\begin{align}
    S^2 = \frac{1}{n}\sum_{i=1}^n \left( X_i - \overline{X} \right)^2 &  &
    S^2 = \frac{1}{\uwave{n-1}}\sum_{i=1}^n \left( X_i - \overline{X} \right)^2
\end{align}
\subsubsection{样本标准差}
\begin{equation}
    S = \sqrt{S^2}
\end{equation}
\subsubsection{样本$k$阶原点矩}
\begin{equation}
    A_k = \frac{1}{n}\sum_{i=1}^n X_i^k,\,\,k\in \ZZ_+
\end{equation}
当$k=1$时, $A_1 = \overline{X}$.
\subsubsection{样本$k$阶中心矩}
\begin{equation}
    B_k = \frac{1}{n}\sum_{i=1}^n \left( X_i - \overline{X} \right)^k,\,\,k\in \ZZ_+
\end{equation}
当$k=2$时, $\frac{n}{n-1}B_2 = S^2$.

\section{常用统计分布}
\leftnote[4cm]{构造性定义}
\subsection{卡方分布}
\dfn{$\chi^2$分布}{
    设$X_1,X_2,\ldots,X_n$是来自$N(0,1)$的样本, 则称随机变量
    \begin{equation}
        \chi^2 = \underbrace{X_1^2 + X_2^2 + \cdots + X_n^2}_{n\text{项}}
    \end{equation}
    服从\vocab{自由度}为$n$的$\chi^2$分布, 记为$\chi^2 \sim \chi^2(n)$.
}
$\chi^2$分布的PDF为
\begin{equation}
    f(x) = \begin{cases}
        \dfrac{1}{2^{n/2}\Gamma\left( \dfrac{n}{2} \right)}x^{n/2-1}e^{-x/2}, & x>0,     \\
        0,                                                                    & x\leq 0.
    \end{cases}
\end{equation}
其中$\Gamma$函数定义为
\begin{equation}
    \Gamma(\alpha) = \int_0^{+\infty} t^{\alpha-1}e^{-t}\dd t,\,\,x>0.
\end{equation}
\qs{例P\textsubscript{123}2}{
    设$X_1,\ldots,X_6$是来自总体$N(0,1)$的样本,又设
    \[Y=(X_1+X_2+X_3)^2 + (X_4 + X_5+X_6)^2,\]
    试求常数$C,\;\st\;CY\sim \chi^2$.
}
\sol{
    因为$X_1 + X_2 +X_3\sim N(0,3),\, X_4+X_5+X_6\sim N(0,3)$,所以
    \begin{align*}
        \dfrac{X_1 + X_2 +X_3}{\sqrt{3}}\sim N(0,1), &  & \dfrac{X_4+ X_5 +X_6}{\sqrt{3}}\sim N(0,1),
    \end{align*}
    因为这两项相互独立,
    \leftnote{定理 (3.2.2)}
    \[\left( \dfrac{X_1 + X_2 + X_3}{\sqrt{3}}^2 \right) + \left(\dfrac{X_4 + X_5 + X_6}{\sqrt{3}}^2 \right)\sim \chi^2(2).\]
    提公共常数$C = \dfrac{1}{3}$即得. 从而$\displaystyle\frac{1}{3}Y\sim\chi^2(2)$.
}
\subsection{t分布}
\dfn{$t$分布}{
    设$X\sim N(0,1),\,Y\sim \chi^2(n)$, 且$X$与$Y$相互独立, 则称随机变量
    \begin{equation}
        t = \frac{X}{\sqrt{Y/n}}
    \end{equation}
    服从自由度为$n$的\vocab{$t$分布}, 记为$t\sim t(n)$.
}
$t$分布的PDF如下. 注意到这是一个偶函数.
\begin{equation}
    f(x) = \frac{\Gamma\left( \dfrac{n+1}{2} \right)}{\sqrt{n\pi}\Gamma\left( \dfrac{n}{2} \right)}\left( 1 + \frac{x^2}{n} \right)^{-\frac{n+1}{2}},\,\,x\in \RR.
\end{equation}
\qs{例P\textsubscript{125}3}{
    设随机变量$X\sim N(2,1)$, 随机变量$Y_1,Y_2,Y_3,Y_4$均服从$N(0,4)$, 且$X,Y_i\;(i=1,2,3,4)$都相互独立,令
    \[T=\dfrac{4(X-2)}{\sqrt{\sum_{i=1}^{4}}Y_i^2},\]
    试求$T$的分布. 并确定$t_0$, 使得$\Pr{\abs{T} > t_0} = 0.01$.
}
\sol{
    由于
    \[X-2\sim N(0,1),\ Y_i/2\sim N(0,1),\ i=1,2,3,4,\]
    根据定义 (5.2.2),
    \[T=\dfrac{4(X-2)}{\sqrt{\displaystyle\sum_{i=1}^{4}Y_i^2}}=\dfrac{X-2}{\sqrt{\displaystyle\sum_{i=1}^{4}\left( \dfrac{Y_i}{4} \right)^2}}=\dfrac{X-2}{\sqrt{\left. \displaystyle\sum_{i=1}^{4}\left( \dfrac{Y_i}{2} \right)^2 \middle/ 4\right.}}\sim t(4)\]
    由$\Pr{\abs{T} > t_0} = 0.01$, 对$n=4,\,\alpha=0.01$,查表得$t_0 = t_{a/2}(4) = t_{0.005}(4) = 4.604$.
}
\exc{P\textsubscript{127}3}{
    设$X,Y$相互独立且都服从$N(0,3^2)$. $X_1,\ldots,X_9$和$Y_1,\ldots,Y_9$是分别取自总体$X,Y$的简单随机样本, 试证明
    \[T=\dfrac{\sum_{i=1}^{9}X_i}{\sqrt{\sum_{i=1}^{9}Y_i^2}}\sim t(9)\]
}
\pf{证明}{
    由题意:  $X_i/3\sim N(0,1),\; Y_i/3\sim N(0,1)\Rightarrow \dfrac{\sum_{i=1}^{9}X_i}{3}\sim N(0,1)\;\;\text{(\circled{1})}$
    \[\text{构造}\sum_{i=1}^{9}Y_i^2\sim \chi^2(9)\tag{\circled{2}}\]
    由\circled{1}\circled{2}构造$T\sim t(9)$, 其中$T$为
    \leftnote{定义 (5.2.2)}
    \begin{align*}
        T=\dfrac{\left. \left( \sum_{i=1}^{9}X_i/3 \right) \middle/ 3\right.}{\sqrt{\left.\sum_{i=1}^{9}\left( Y_i/3 \right)^2\middle/ 9\right.}} &  & \text{化简有}\ \ = \dfrac{\sum_{i=1}^{9}X_i}{\sqrt{\sum_{i=1}^{9}Y_i^2}}
    \end{align*}
}
\subsection{分位数}
\dfn{分位数}{
    设$X$是一个连续型随机变量, $0<\alpha<1$, 若实数$x_\alpha$满足
    \[\Pr{X< x_\alpha} = \alpha,\]
    则称$x_\alpha$为$X$的水平为$p$的\vocab{上侧分位数}.
}
若实数$x_{\alpha/2}$满足
\[\Pr{\abs{X}>x_{\alpha/2}} = x_{\alpha/2},\]
则称$x_{\alpha/2}$为$X$的水平为$\alpha$的\vocab{双侧分位数}. 一般不直接求解分位数, 对常用的统计分布一般查表来得到分位数的值.

\textbf{注}\quad 记$u_\alpha$为标准正态分布的水平为$\alpha$的上侧分位数, $\chi_\alpha^2(n)$为自由度为$n$的$\chi^2$分布的水平为$\alpha$的上侧分位数, $t_\alpha(n)$为自由度为$n$的$t$分布的水平为$\alpha$的上侧分位数. 有结论 (利用PDF是偶函数的对称性)
\begin{equation}
    u_\alpha = -u_{1-\alpha},\quad t_\alpha(n) = -t_{1-\alpha}(n)
\end{equation}
\ex{P\textsubscript{121}例1}{
    设$\alpha=0.05$, 求标准正态分布的水平$0.05$的上侧分位数和双侧分位数.
}
\sol{
    由于$\Phi(u_{0.05}) = 1 - 0.05 = 0.95$, 查表有$u_{0.05} = 1.645$.
    双侧分位数$u_{0.025}$满足$\Phi(u_{0.025}) = 1 - 0.025 = 0.975$, 查表有$u_{0.025} = 1.96$.\\
}
\subsection{F分布}
\dfn{$F$分布}{
    设$X\sim \chi^2(m),\,Y\sim \chi^2(n)$, 且$X$与$Y$相互独立, 则称随机变量
    \begin{equation}
        F = \frac{X/m}{Y/n}
    \end{equation}
    服从自由度为$(m,n)$的\vocab{$F$分布}, 记为$F\sim F(m,n)$.
}
若$X\sim t(n)$, 则$X^2\sim F(1,n)$; 若$F\sim F(m,n)$, 则$\frac{1}{F}\sim F(n,m)$.

记$F$分布的分位数为$F_{\alpha}(m,n)$, 其中$a,b$分别为分位数的上侧概率和下侧概率.
\qs{证明}{
    \begin{equation}
        F_\alpha(m,n) = \dfrac{1}{F_{1-\alpha}(n,m)}.
    \end{equation}
}
\pf{证明}{
设$F\sim F(m,n)$, 移项有
\leftnote[1cm]{定义 (5.2.3)}
\[\Pr{F< F_\alpha(m,n)} = \alpha = \Pr{\dfrac{1}{F}\geq \dfrac{1}{F_\alpha(m,n)}}.\]
将$1/F$整体看作一个随机变量,
\leftnote{定义 (2.3.1)}
\[1-\alpha = \Pr{\dfrac{1}{F}< \dfrac{1}{F_\alpha(m,n)}}\tag{$\ast$}\]
根据$F$分布的性质, $1/F\sim F(n,m)$, 其水平为$1-\alpha$的分位数$F_{1-\alpha}(n,m)$满足
\[1-\alpha = \Pr{\dfrac{1}{F} < F_{1-\alpha}(n,m)}\]
与 ($\ast$)式比较,
\[\tf\;\;F_{1-\alpha}(n,m) = \dfrac{1}{F_\alpha(m,n)}\Longrightarrow F_\alpha(m,n) = \dfrac{1}{F_{1-\alpha}(n,m)}.\]
}

\section{抽样分布}
\subsection{单正态总体的抽样分布}
\thm{}{
    设总体$X\sim N(\mu,\sigma)$, $X_1,\ldots,X_n$是取自$X$的一个样本, $\overline{X}$和$S^2$分别是样本均值和样本方差, 则
    \begin{align}
         & (1)\;       \overline{X}\sim N(\mu,\sigma^2/n);                                                                          \\
         & (2^\ast)\;  U = \dfrac{\overline{X} - \mu}{\sigma/\sqrt{n}}\sim N(0,1);                                                  \\
         & (3^\ast)\;  \chi^2 = \dfrac{n-1}{\sigma^2}S^2 = \dfrac{1}{\sigma^2}\sum_{i=1}^{n}(X_i - \overline{X})^2\sim \chi^2(n-1); \\
         & (4)\;       \overline{X}\text{与}S^2\text{相互独立}.                                                                          \\
         & (5)\;       \chi^2 = \dfrac{1}{\sigma^2}\sum_{i=1}^{n}(X_i-\mu)^2\sim \chi^2(n);                                         \\
         & (6^\ast)\;  T = \dfrac{\overline{X} -\mu}{S/\sqrt{n}}\sim t(n-1)
    \end{align}
}
标$\ast$号的是后面常用的定理.

一些常用的记号:
\begin{itemize}
    \item[总体] $\mu$\quad 总体均值
    \item $\sigma$\quad 总体\emph{标准差}
    \item[样本] $\overline{X}$ \quad 样本均值
    \item $S$\quad 样本\emph{标准差}
\end{itemize}
