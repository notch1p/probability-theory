\chapter{假设检验}
假设检验就是反证法. 为了验证某个假设是否正确, 先使其成立, 根据抽取的样本计算出一个统计量 (方差, 均值), 若该统计量的值落在一个\uwave{不合理}的范围 (拒绝域)内, 则认为不成立, 反之成立.
\section{基本概念}
上述的\emph{不合理}可以表示为小概率事件. 小概率事件发生的概率越小, 说明假设成立的可能性越小, 因此拒绝假设的理由越充分. 但是, 不能因为拒绝域内的事件发生了就拒绝假设 (如果拒绝域十分小, 仍然可以接受假说), 因为这样会导致\vocab{第一类错误}. 犯这个错误的概率就是小概率事件发生的概率,即
\leftnote{考虑法庭判决: \uwave{判错无辜的人}, 这就是第一类错误.}
\[
    \Pr{\text{reject}\ H_0 | H_0\ \text{is true}} = \alpha
\]
为了避免第一类错误, 需要选择一个合适的拒绝域, 使得第一类错误的概率不超过预先规定的值$\alpha$, 这个值称为\vocab{显著性水平} (就是上侧分位数). 一般取$\alpha = 0.05$或$\alpha = 0.01$. 我们把要检验的假设$H_0$称为\vocab{原假设}, 因此有\vocab{对立假设}$H_1$存在. 例如:
\begin{align*}
    H_0:\mu = \mu_0; &  & H_1:\mu \neq \mu_0.
\end{align*}
这称为双侧假设检验.

经常需要构造\uwave{检验统计量} (一般记为$U, T, \chi$), 当检验统计量的值落在某个区域$W$内时, 拒绝原假设, 称$W$为\vocab{拒绝域}. 一般来说, 置信区间外的值都是拒绝域内的值. 显著性水平正好和置信水平相反.
\section{单正态总体的假设检验}
\subsection{总体均值的假设检验}
\uwave{若是检验总体均值, 最后就代入样本均值; 若是检验总体方差, 最后就代入样本方差.}
\subsubsection{方差已知}
与前面求置信区间类似, 设总体$X\sim N(\mu, \sigma^2)$. 总体方差$\sigma^2$已知. $X_1,\ldots,X_n$是取自总体$X$的样本, 给定显著性水平$\alpha$, 检验假设$H_0:\mu=\mu_0$, $H_1:\mu\neq\mu_0$. \uwave{以$H_0$成立为前提}, 构造检验统计量
\leftnote{定理 (5.3.1)-2, 就是构造标准正态分布}
\begin{equation}
    U = \dfrac{\overline{X} - \mu_0}{\sigma/\sqrt{n}}\sim N(0,1)
\end{equation}
记观察值为$u$. 这称为\vocab{$u$检验法}.

根据拒绝域的定义, 可以得到
\begin{equation}
    W = (-\infty, -u_{\alpha/2}) \cup (u_{\alpha/2}, +\infty).
\end{equation}
根据样本计算样本均值$\overline{x}$, 将$\mu_0$, $\sigma$代入, 计算$u$, 若$u\in W$, 则拒绝$H_0$, 否则接受$H_0$.
\ex{P\textsubscript{165}例1}{
    设总体$X\sim N(500, 2^2)$. 取其中的$9$个样本如下:
    \begin{align*}
        505 &  & 499 &  & 502 &  & 506 &  & 498 &  & 498 &  & 497 &  & 510 &  & 503
    \end{align*}
    \uwave{总体标准差不变}. 检验假设$H_0:\mu = 500$, $H_1:\mu\neq 500$. ($\alpha = 0.05$)
}
\sol{
以$H_0$成立为前提, 构造检验统计量
\[U = \dfrac{\overline{X} - \mu_0}{\sigma/\sqrt{n}} = \dfrac{\overline{X} - 500}{2/3}\sim N(0,1)\]
将样本均值$\overline{x} = 502$代入, 计算观察值得$u = 3$.
\[\tf\;u\in W = (-\infty, -u_{\alpha/2})\cup (u_{\alpha/2}, +\infty) = (-1.96, -\infty)\cup (1.96, +\infty)\]
因此不接受$H_0$.
}
\subsubsection{方差未知}
题设同上, 但方差未知. 同理可以构造检验统计量$T$, 因为$\overline{X}$为$\mu$的\uwave{无偏估计量}, $S^2$是$\sigma^2$的\uwave{无偏估计量},
\leftnote{定理 (5.3.1)-6}
我们用样本方差 (计算无偏方差)来代替总体方差, 以$H_0$成立为前提,
\begin{equation}
    T = \dfrac{\overline{X} - \mu_0}{S / \sqrt{n}}\sim t(n-1)
\end{equation}
记观察值为$t$. 这称为\vocab{$t$检验法}.

根据拒绝域的定义, 可以得到 ($t$分布的PDF是偶函数)
\begin{equation}
    W = (-\infty, -t_{\alpha/2}(n-1)) \cup (t_{\alpha/2}(n-1), +\infty).
\end{equation}
将样本均值$\overline{x}$, 样本方差$s^2$代入, 计算$t$, 若$t\in W$, 则拒绝$H_0$, 否则接受$H_0$.
\subsection{总体方差的假设检验}
设$X\sim N(\mu,\sigma^2)$. $X_1,\ldots,X_n$是取自$X$的样本, 给定显著性水平$\alpha$, 检验假设$H_0:\sigma^2=\sigma_0^2$, $H_1:\sigma^2\neq\sigma_0^2$. 以$H_0$成立为前提, 构造检验统计量$\chi^2$
\leftnote{定理 (5.3.1)-3}
\begin{equation}
    \chi^2 = \frac{n-1}{\sigma_0^2}S^2\sim\chi^2(n-1),
\end{equation}
由$\chi^2$分布的PDF图象可以知道, 两侧分别为$\chi^2_{1-\alpha/2}(n-1)$, $\chi^2_{\alpha/2}(n-1)$, 记观察值为$\chi^2$, 这称为\vocab{$\chi^2$检验法}.
因此
\begin{equation}
    W = [0, \chi^2_{\alpha/2}(n-1)) \cup (\chi^2_{1-\alpha/2}(n-1), +\infty).
\end{equation}
将样本方差$s^2$代入, 计算$\chi^2$, 若$\chi^2\in W$, 则拒绝$H_0$, 否则接受$H_0$.
\exc{P\textsubscript{171}5}{
    某种导线的电阻服从正态分布$N(\mu, 0.005^2)$. 从一批导线中抽取$9$根, 测定电阻, 得$s = 0.008$. 对$\alpha=0.05$, 能否认为这批导线电阻的标准差仍为$0.005$?
}
\sol{
设导线的电阻为$X$. 由题意$X\sim N(\mu, 0.005^2)$.
设$H_0: \sigma = 0.005$, $H_1: \sigma\neq 0.005$. 在$H_0$成立的前提下,构造
\[\chi^2 = \frac{n-1}{0.005^2}s^2\sim\chi^2(8)\]
因为$\alpha = 0.05, \alpha/2 = 0.025$, $\chi^2_{0.025}(8) = 17.535$, $\chi^2_{0.975}(8)= 2.180$, 因此
\[W = [0,2.180)\cup(17.54,+\infty)\]
计算观察值,
\[\chi^2 = \frac{8}{0.005^2}\times 0.008^2 = 20.48\in W\]
因而不接受假说$H_0$. 不能认为这批导线电阻的标准差仍为$0.005$.
}