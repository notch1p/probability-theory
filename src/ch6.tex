\chapter{参数估计}
\section{点估计问题概述}
\subsection{点估计的概念}
设$X_1,\ldots,X_n$是取自总体$X$的一个样本, $x_1,\ldots,x_n$是相应的一个样本值,$\theta$是总体分布中的未知参数, 为估计未知参数$\theta$, 需要构造一个适当的统计量
\leftnote{$\hat{\theta}$是随机变量, 也是样本的函数}
\begin{equation}
    \hat{\theta}(X_1,\ldots,X_n)
\end{equation}
并用其观察值
\begin{equation}
    \hat{\theta}(x_1,\ldots,x_n)
\end{equation}
来估计$\theta$的值. 式 (6.1)和式 (6.2)分别称为\vocab{点估计\uwave{量}}和\vocab{点估计\uwave{值}}.
\subsection{评价估计的标准}
\subsubsection{无偏性}
\dfn{}{
    设$\hat{\theta}$是$\theta$的一个估计量, 若
    \begin{equation}
        \EE{\hat{\theta}} = \theta,
    \end{equation}
    则称$\hat{\theta}$是$\theta$的一个\vocab{无偏估计量}, 否则称为\vocab{有偏估计量}.
}
将$\EE{\hat{\theta}}-\theta$称为用$\hat{\theta}$估计$\theta$的系统偏差. 对一般总体而言, 有
\thm{Slutsky's theorem}{
    设$X_1,\ldots,X_n$是来自总体$X$的样本, 总体$X$的均值为$\mu$, 方差为$\sigma^2$, 则
    \begin{enumerate}
        \item 样本均值$\overline{X}$是$\mu$的无偏估计量;
        \item 样本方差$S^2$是$\sigma^2$的无偏估计量.
        \item $B_2=\frac{1}{n}\sum_{i=1}^{n}(X_i - \overline{X})^2$是$\sigma^2$的\emph{渐进无偏} ($n\to\infty$)估计量.
    \end{enumerate}
}
\thm{}{
    若$\sum_{i}a_i = 1$, 则估计量
    \[
        \begin{pmatrix}
            a_1 & \cdots & a_n
        \end{pmatrix}\begin{pmatrix}
            X_1    \\
            \vdots \\
            X_n
        \end{pmatrix}
    \]
    是$\mu$的无偏估计量.
}
\exc{P\textsubscript{140}3}{
    设$\hat{\theta}$是参数$\theta$的无偏估计. 且$\Var{\hat{\theta}} > 0$, 试证$\hat{\theta}^2 = \left(\hat{\theta}\right)^2$不是$\theta^2$的无偏估计.
}
\pf{Proof}{
    \begin{align*}
        \bc & \Var{\hat{\theta}}                           > 0, \\
        \leftnote{\eqref{eq:4.9}}
        \tf & \EE{\hat{\theta}^2} - \EE[2]{\hat{\theta}}  > 0.
        \intertext{因为$\EE{\hat{\theta}} = \theta$, 移项有}
        \tf & \EE{\hat{\theta}^2} > \theta^2.
    \end{align*}
}
\subsubsection{有效性}
在没有系统偏差的情况下, 比较随机误差.
\dfn{}{
    设$\hat{\theta}_1$和$\hat{\theta}_2$是$\theta$的两个无偏估计量, 有
    \begin{equation}
        \Var{\hat{\theta}_1} \leq \Var{\hat{\theta}_2},
    \end{equation}
    则称$\hat{\theta}_1$比$\hat{\theta}_2$\vocab{有效}.
}
\subsubsection{相合性}
\dfn{}{
    设$\hat{\theta} = \hat{\theta}(X_1,\ldots,X_n)$为未知参数$\theta$的估计量. 若$\hat{\theta}\xrightarrow{P}\theta,\,\ie\ \epsilon>0$,
    \begin{equation}
        \lim_{n\to\infty}\Pr{\abs{\hat{\theta}_n - \theta}<\epsilon} = 1,
    \end{equation}
    则称$\hat{\theta}_n$是$\theta$的一个\vocab{相合估计量}.
}
\section{点估计的常用方法}
\subsection{矩估计法}
由大数定律我们知道,当总体的$k$阶矩存在时, 样本的$k$阶矩依概率收敛于总体的$k$阶矩. 譬如说可用样本均值$\overline{X}$作为总体均值$\EE{X}$的估计量. 一般地,记
\begin{enumerate}
    \item 总体$k$阶原点矩 $\mu_k=\EE{X^k}$
    \item 样本$k$阶原点矩 $A_k=\frac{1}{n}\sum_{i=1}^nX_i^k$
    \item 总体$k$阶中心矩 $\nu_k=\EE[k]{X-\EE{X}}$
    \item 样本$k$阶中心矩 $B_k=\frac{1}{n}\sum_{i=1}^n(X_i-\overline{X})^k$
\end{enumerate}
\ex{P\textsubscript{141}例3}{
    设总体$X$的概率分布为
    \begin{center}
        \begin{tabular}{@{}llll@{}}
            \toprule
            $X$   & $1$        & $2$                 & $3$            \\ \midrule
            $p_i$ & $\theta^2$ & $2\theta(1-\theta)$ & $(1-\theta)^2$ \\ \bottomrule
        \end{tabular}
    \end{center}
    其中$0<\theta<1$. 现抽得一个样本$x_1 = 1,\,x_2 = 2\,x_3=1$, 求$\theta$的矩估计量.
}
\sol{
    \[\mu_1 = \EE{X} = \theta^2 + 2\times 2\theta(1-\theta) +3(1-\theta)^2 = 3-2\theta,\]
    \[A_1 = \dfrac{1}{3}(1+2+1) = \dfrac{4}{3}\]
    \emph{令之相等}, 得$3-2\theta = \dfrac{4}{3}$, 即$\theta$矩估计$\hat{\theta} = \dfrac{5}{6}$.
    \leftnote{体现矩估计思想: 样本矩等于总体矩}
}
\subsection{最大似然估计法}
\dfn{}{
    对任意给定样本值$x_1,\ldots,x_n$,
    \[\exists\hat{\theta}=\hat{\theta}(\bmx),\,\text{where}\ \uwave{\bmx=(x_1,\ldots,x_n)}\;\st\;
        \LL(\hat{\theta}) = \argmax_\theta\LL(\theta;\bmx),
    \]
    则称$\hat{\theta} = \hat{\theta}(\bmx)$为$\theta$的\vocab{最大似然估计值}, $\hat{\theta}(X_1,X_2,\ldots,X_n)$为$\theta$的\vocab{最大似然估计量}.
    统称为\vocab{最大似然估计 (MLE)}.
}
其中$\LL(\theta)$称为样本的\vocab{似然函数}.
若为\emph{离散型}, 设总体$X$的概率分布为$\Pr{X=x}=p(x;\theta)$, 如果$X_1,X_2,\ldots,X_n$是来自$X$的样本, 观察值为$\bmx$, 记$\LL(\theta)$:
\begin{equation}
    \LL(\theta) = \LL(\underbrace{\bmx}_{\text{参数}};\overbrace{\theta}^{\text{变量}}) = \prod_{i=1}^n p(x_i;\theta)
\end{equation}

若为\emph{连续型}, 设总体$X$的PDF为$f(x;\theta)$, $X_1,X_2,\ldots,X_n$是来自$X$的样本 ($n$维连续型随机变量, 联合PDF为$f(\bmx)$), 观察值为$\bmx$.
因为$(X_1,\ldots,X_n)$\emph{相互独立}, 根据\eqref{eq:3.15},
\[f(x_1,\ldots,x_n) = f_{x_1}(x_1)f_{x_2}(x_2)\cdots f_{x_n}(x_n),\]
因此记$\LL(\theta)$:
\begin{equation}
    \LL(\theta) = \LL(\bmx;\theta) = \prod_{i=1}^n f(x_i;\theta)
\end{equation}

为了方便计算, 经常在两侧取对数, 称为\vocab{对数似然函数}:
\begin{equation}
    \ell(\bmx;\theta) = \ln{\LL(\bmx;\theta)}
\end{equation}
因此, 只要求出一个$\theta$\;\st\;$\LL(\theta)$取最大值即可, 记这个$\theta$为$\hat{\theta}$, 对可微函数$\LL(\theta)$,
\leftnote{对似然函数而言, 驻点即为极大值点.}
令$\LL'(\theta) = 0$或$\ell'(\theta) = 0$, 求出驻点.
\ex{P\textsubscript{143}例4}{
    设$X\sim b(1,p)$, $X_1,X_2,\ldots,X_n$是来自$X$的样本, 求$p$的MLE.
}
\sol{
设$x_1,\ldots,x_n$是相应于样本$X_1,\ldots,X_n$的观察值, $X$的分布律为
\[\Pr{X=x} = p^x(1-p)^{1-x},\;\;x=0,1.\]
于是似然函数为
\[\LL(p) = p^{\sum_{1}^{n}x_i}\times (1-p)^{\sum_{1}^{n}1-x_i},\]
取对数
\[\ln{\LL(p)} = \left(\sum_{i=1}^{n}x_i\right)\ln{p} + \left(n-\sum_{i=1}^{n}x_i\right)\ln{(1-p)},\]
求导
\[\dv{\ln{\LL(p)}}{p} = \dfrac{\sum_{i = 1}^{n}x_i}{p} - \dfrac{n - \sum_{i=1}^{n}x_i}{1-p} = 0\]
解出$\hat{p}(x_1,\ldots,x_n)$
\leftnote{最大似然\emph{估计值}}
\[\hat{p} = \dfrac{\sum_{i=1}^{n}x_i}{n} = \overline{x},\]
可得$\hat{p}(X_1,\ldots,X_n)$
\leftnote{最大似然\emph{估计量}}
\[\hat{p} = \dfrac{\sum_{i=1}^{n}X_i}{n} = \overline{X},\]
}
\ex{P\textsubscript{144}例5}{
    设$X\sim e(\lambda)$, $X_1,X_2,\ldots,X_n$是来自$X$的样本, 求$\lambda$的MLE.
}
\sol{
    似然函数为
    \begin{align*}
        \LL(x_1,\ldots,x_n; \lambda)                    = \prod_{i=1}^{n}f_{x_i}(x_i)
                                                       & =\prod_{i = 1}^{n}\lambda^n e^{-\lambda x_i} \\
                                                       & =\lambda^n e^{-\lambda\sum_{i=1}^{n}x_i},
        \intertext{取对数}
        \ln{\LL(x_1,\ldots,x_n;\lambda)}               & =n\ln{\lambda}-\lambda\sum_{i=1}^{n}x_i,
        \intertext{求导}
        \dv{\ln{\LL(x_1,\ldots,x_n;\lambda)}}{\lambda} & =\dfrac{n}{\lambda}-\sum_{i=1}^{n}x_i=0,
    \end{align*}
    解得$\lambda$的最大似然估计值$\displaystyle\hat{\lambda}(x_1,\ldots,x_n) = \dfrac{n}{\sum_{i=1}^{n}x_i} = \dfrac{1}{\overline{x}}$
}
\section{置信区间}
\subsection{置信区间的概念}
设$\theta$是总体$X$的未知参数, $\hat{\theta}_1(X_1,\ldots,X_n)$和$\hat{\theta}_2(X_1,\ldots,X_n)$是$\theta$的两个估计量, 如果对任意给定的$\alpha>0$, 有
\begin{equation}
    \Pr{\hat{\theta}_1<\theta<\hat{\theta}_2}= 1-\alpha,
\end{equation}
则称随机区间
\[\hat{\theta}_1<\theta<\hat{\theta}_2\]
是$\theta$的一个\vocab{置信度}为$1-\alpha$的\vocab{置信区间}, $\hat{\theta}_1$和$\hat{\theta}_2$分别称为$\theta$的\vocab{置信下限}和\vocab{置信上限}.
\leftnote{要求置信区间尽可能短}
在保证置信度的条件下, 尽可能提高估计精度.
\section{正态总体的置信区间}
\subsection{正态总体\uwave{均值}的置信区间}
设$X_1,\ldots,X_n$是来自正态总体$N(\mu,\sigma^2)$的样本, $\mu$未知而$\sigma$已知, 对给定的置信水平$1-\alpha$, 由标准正态分布PDF的\uwave{对称性}可得$\mu$的置信区间为
\begin{equation}
    \left(\overline{X} - u_{\alpha/2}\times\dfrac{\sigma}{\sqrt{n}},\, \overline{X}+u_{\alpha/2}\times\dfrac{\sigma}{\sqrt{n}}\right).
    \label{eq:6.6}
\end{equation}
其中$u_{\alpha/2}$是标准正态分布的上侧$\alpha/2$分位数.
\ex{P\textsubscript{151}例1}{
设$X_1,\ldots,X_{100}$是来自正态总体的样本, $\overline{x}=80,\sigma = 12$, 求$\mu$的置信度为$0.95$的置信区间.
}
\sol{
    \leftnote{\eqref{eq:6.6}}
    由题意$1-\alpha = 0.95\Rightarrow\alpha = 0.05$, 由标准正态分布表可得$u_{\alpha/2} = u_{0.025} = 1.96$, 代入有区间$(77.6,82.4)$.
}

设$X_1,\ldots,X_n$是来自正态总体$N(\mu,\sigma^2)$的样本, $\mu,\sigma$\emph{都未知}, 对给定的置信水平$1-\alpha$, 用$\sigma^2$的无偏估计$S^2$代替, 构造枢轴变量
\leftnote{定理 (6.1.1)-2}
\begin{equation*}
    T = \dfrac{\overline{X}-\mu}{S/\sqrt{n}},
\end{equation*}
\leftnote{定理 (5.3.1)-6}
因为$T\sim t(n-1)$, 由
\[\Pr{-t_{\alpha/2}(n-1)<\dfrac{\overline{X} - \mu}{S/\sqrt{n}}<t_{a/2}(n-1)} = 1-\alpha,\]
得$\mu$的置信区间为
\begin{equation}
    \left(\overline{X}-t_{\alpha/2}(n-1)\times\dfrac{S}{\sqrt{n}},\,\overline{X}+t_{\alpha/2}(n-1)\times\dfrac{S}{\sqrt{n}}\right).
\end{equation}
\ex{P\textsubscript{151}例2}{
设$X_1,\ldots,X_{25}$是来自正态总体的样本, $\overline{x}=80$, 样本标准差$s = 12$, 求$\mu$的置信度为$0.95$的置信区间.
}
\sol{
根据题意, $t_{\alpha/2}(n - 1)=t_{0.025}(24)=2.0639$. 将$\sigma$用$s$代替, 代入有区间$(75.05,84.95)$.
}
\exc{P\textsubscript{161}18}{
    某公司生产电池, 现从其产品中抽取$50$只电池做寿命试验, 这些电池的平均寿命$\overline{x} = 2.266$, $s = 1.935$. 求电池平均寿命的置信度为$95\%$的置信区间.
}
\sol{
设电池寿命为$X$. 由题意构造枢轴变量
\[T = \frac{\overline{X} - \mu}{S / \sqrt{n}},\]
\[-t_{\alpha/2}(n-1)<\dfrac{\overline{X} - \mu}{S/\sqrt{n}}<t_{a/2}(n-1).\]
可得$\mu$的置信区间为
\begin{equation*}
    \left(\overline{X}-t_{\alpha/2}(n-1)\times\dfrac{S}{\sqrt{n}},\,\overline{X}+t_{\alpha/2}(n-1)\times\dfrac{S}{\sqrt{n}}\right),
\end{equation*}
将$t_{0.025}(49) = 1.96$, $\overline{x} = 2.266$, $s = 1.935$, $n = 49$代入可得$(1.730, 2.802)$.
}
\subsection{正态总体方差的置信区间}
设$X_1,\ldots,X_n$是来自正态总体$N(\mu,\sigma^2)$的样本, $\mu,\sigma$\emph{都未知}, 对给定的置信水平$1-\alpha$, 构造枢轴变量
\leftnote{定理 (5.3.1)-3}
\[
    \dfrac{n-1}{\sigma^2}S^2\sim \chi^2(n-1)
\]
因为卡方分布的PDF只在正半轴上有值 (\uwave{不具有对称性}), 所以两侧为$\chi^2_{1-\alpha/2}(n-1)$, $\chi^2_{\alpha/2}(n-1)$, 且
\[
    \Pr{\chi^2_{1-\alpha/2}(n-1)< \dfrac{n-1}{\sigma^2}S^2<\chi^2_{\alpha/2}(n-1)} = 1 -\alpha
\]
求解$\sigma^2$, 得方差为$\sigma^2$的$1-\alpha$置信区间为
\begin{equation}
    \left(\dfrac{(n-1)S^2}{\chi^2_{1-\alpha/2}(n-1)},\,\dfrac{(n-1)S^2}{\chi^2_{\alpha/2}(n-1)}\right).
\end{equation}
对上式开根号即得标准差为$\sigma$的$1-\alpha$置信区间.
